%!TEX root=../master_thesis.tex
\chapter{Conclusion}
In this thesis we gave an overview on the inner workings of \ac{RCL} and its Consistent Hashing implementation.
Inspired by Fritchie's idea to replace the Consistent Hashing algorithm with \ac{RS} we also presented how \ac{RS} works.
Seeing that Consistent Hashing directly allows for a simple and quick \ac{RPS} we also presented a choice of simple \acp{RPS}.
We then implemented a prototype version of \ac{RCL} with \ac{RS} as its data partitioning algorithm and gave an overview of how the structure and some workflows of \ac{RCL} had to be changed.
To compare both versions of \ac{RCL} we designed configurations to illustrate the strengths and weaknesses of both systems.
With those configuration we ran benchmarks to retrieve data.
From that data we created a first impression of how \ac{RS} works with \ac{RCL} and what the big problems are with the current implementation.

This whole process has shown that the choice of the partitioning algorithm in \ac{RCL} is rather an architectural choice than a design one as large parts of the system depend on the inner workings of this algorithm.
In turn, simply replacing the partitioning component and redesigning components directly connected to it is not enough.
One has to rather go back to the architectural phase and reevaluate the design decisions depending on the chosen partitioning algorithm and its requirements, guarantees and other attributes.
Considering \ac{RS} was adopted for \ac{RCL} in a prototype manner without a full-scope design process the evaluation results look promising.
For the tested aspects it produced metrics that were at most slightly worse than \ac{RCL} with Consistent Hashing.
Especially problems during handoff periods and with load balancing can be clearly overcome by a more refined development process in the future.
Overall, we showed that \ac{RS} works as a partitioning algorithm, though it is reliant on its actual gap collection algorithm and replica placement strategy.
We could not show all of the theoretical improvements listed by Fritchie, especially the improvements of the handoff overhead.
However, this was mostly caused by the prototype implementation and missing metrics in the evaluation.
We strongly believe that with a full redesign of affected components and considering our proposed future work \ac{RS} can improve \ac{RCL}'s overall performance and especially its scalability and adaptability in more dynamic clusters.