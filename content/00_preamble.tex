%!TEX root = ../master_thesis.tex

\documentclass[a4paper,11pt]{scrreprt}

\usepackage[utf8]{inputenc}
\usepackage[T1]{fontenc}
\usepackage[ngerman, english]{babel}
\usepackage[hidelinks=true]{hyperref}
\usepackage{acronym}
\usepackage[backend=biber]{biblatex}
\usepackage{amsmath,amsfonts}
\usepackage{listings}
\usepackage[linesnumbered]{algorithm2e}
\usepackage{graphicx}
\usepackage{tabularx}
\usepackage{glossaries}
\usepackage{enumitem}
\usepackage{xcolor}
\usepackage{xspace}
\usepackage{adjustbox,storebox}
\usepackage{newfloat}
\usepackage{caption}
\usepackage{subcaption}
\usepackage{csvsimple}
\usepackage[capitalize,noabbrev]{cleveref}
\usepackage[multiple]{footmisc}


\definecolor{commentsColor}{rgb}{0.497495, 0.497587, 0.497464}
\definecolor{keywordsColor}{rgb}{0.000000, 0.000000, 0.635294}
\definecolor{stringColor}{rgb}{0.558215, 0.000000, 0.135316}

\graphicspath{{./images/}}

\addbibresource{master_thesis.bib}

\newcommand*{\MyTitle}{Adopting Random Slicing for Riak Core}
\newcommand*{\GermanTitle}{Die Übernahme von Random Slicing für Riak Core}

\newcommand*{\call}[2][]{
	\item #1 %\lstinline!#2!
}

\newcommand{\rrangle}{>\kern-1.2ex~>\xspace}
\newcommand{\llangle}{<\kern-1.2ex~<\xspace}
\newcommand*{\actiontag}[1]{$\llangle$\textit{#1}$\rrangle$}
\newcommand*{\actioninfo}[1]{$\lbrack$#1$\rbrack$}

\newcommand{\evaluation}[2]{
\begin{figure}
\begin{subfigure}{0.5\textwidth}
	\includegraphics[width=0.8\textwidth]{#1_#2_read_heavy_throughput}
	\caption[Throughput for \lstinline!#1_#2_read_heavy!]{Throughput for \lstinline!#1_#2_read_heavy!}
\end{subfigure}
\begin{subfigure}{0.5\textwidth}
	\includegraphics[width=0.8\textwidth]{#1_#2_read_heavy_latency}
	\caption[Latency for \lstinline!#1_#2_read_heavy!]{Latency for \lstinline!#1_#2_read_heavy!}
\end{subfigure}
\begin{subfigure}{0.5\textwidth}
	\includegraphics[width=0.8\textwidth]{#1_#2_write_heavy_throughput}
	\caption[Throughput for \lstinline!#1_#2_write_heavy!]{Throughput for \lstinline!#1_#2_write_heavy!}
\end{subfigure}
\begin{subfigure}{0.5\textwidth}
	\includegraphics[width=0.8\textwidth]{#1_#2_write_heavy_latency}
	\caption[Latency for \lstinline!#1_#2_write_heavy!]{Latency for \lstinline!#1_#2_write_heavy!}
\end{subfigure}
\begin{subtable}{\textwidth}
	\tiny
	\center
	\csvautotabular[/csv/respect underscore=true]{./EvaluationData/#1_#2_read_heavy/#1_#2_read_heavy_stats.csv}
	\caption[Stats for \lstinline!#1_#2_read_heavy!]{Stats for \lstinline!#1_#2_read_heavy!}
\end{subtable}
\begin{subtable}{\textwidth}
	\tiny
	\center
	\csvautotabular[/csv/respect underscore=true]{./EvaluationData/#1_#2_write_heavy/#1_#2_write_heavy_stats.csv}
	\caption[Stats for \lstinline!#1_#2_write_heavy!]{Stats for \lstinline!#1_#2_write_heavy!}
\end{subtable}
\caption[Evaluation Data for \lstinline!#1_#2!]{Evaluation Data for \lstinline!#1_#2!}
\end{figure}
}

\makenoidxglossaries

\newglossaryentry{cloud computing}{
name={cloud computing},
description={is the usage of a set of connected network computing devices called \emph{cloud} to perform a shared task.
The }
}

\newglossaryentry{cluster}{
name={cluster},
description={is a collection of potentially distributed entities that store and process data}
}

\newglossaryentry{node}{
name={node},
description={is an entity that can store and process data and communicate with other nodes.
	A node is called \emph{physical node} if it is represented by one or more \emph{virtual nodes}.
	A \emph{virtual node} is part of exactly one physical node
}
}

\newglossaryentry{section}{
name={section},
description={is a partition of a space.
Sections are ordered by their starting index}
}

\lstset{ %
  backgroundcolor=\color{white},   % choose the background color; you must add \usepackage{color} or \usepackage{xcolor}
  basicstyle=\footnotesize,        % the size of the fonts that are used for the code
  breakatwhitespace=false,         % sets if automatic breaks should only happen at whitespace
  breaklines=true,                 % sets automatic line breaking
  captionpos=b,                    % sets the caption-position to bottom
  commentstyle=\color{commentsColor}\textit,    % comment style
  extendedchars=true,              % lets you use non-ASCII characters; for 8-bits encodings only, does not work with UTF-8
  keepspaces=true,                 % keeps spaces in text, useful for keeping indentation of code (possibly needs columns=flexible)
  keywordstyle=\color{keywordsColor}\bfseries,       % keyword style
  language=Python,                 % the language of the code (can be overrided per snippet)
  otherkeywords={*,...},           % if you want to add more keywords to the set
  numbers=left,                    % where to put the line-numbers; possible values are (none, left, right)
  numbersep=5pt,                   % how far the line-numbers are from the code
  numberstyle=\tiny\color{commentsColor}, % the style that is used for the line-numbers
  rulecolor=\color{black},         % if not set, the frame-color may be changed on line-breaks within not-black text (e.g. comments (green here))
  showspaces=false,                % show spaces everywhere adding particular underscores; it overrides 'showstringspaces'
  showstringspaces=false,          % underline spaces within strings only
  showtabs=false,                  % show tabs within strings adding particular underscores
  stepnumber=1,                    % the step between two line-numbers. If it's 1, each line will be numbered
  stringstyle=\color{stringColor}, % string literal style
  tabsize=2,	                   % sets default tabsize to 2 spaces
  title=\lstname,                  % show the filename of files included with \lstinputlisting; also try caption instead of title
  columns=fixed                    % Using fixed column width (for e.g. nice alignment)
}

\setlistdepth{11}

\newlist{action}{enumerate}{11}
\setlist[action,1]{label=\arabic*}
\setlist[action,2]{label=\labelactioni.\arabic*}
\setlist[action,3]{label=\labelactionii.\arabic*}
\setlist[action,4]{label=\labelactioniii.\arabic*}
\setlist[action,5]{label=\labelactioniv.\arabic*}
\setlist[action,6]{label=\labelactionv.\arabic*}
\setlist[action,7]{label=\labelactionvi.\arabic*}
\setlist[action,8]{label=\labelactionvii.\arabic*}
\setlist[action,9]{label=\labelactionviii.\arabic*}
\setlist[action,10]{label=\labelactionix.\arabic*}
\setlist[action,11]{label=\labelactionx.\arabic*}


\DeclareFloatingEnvironment[
fileext=loc,
listname={List of actions},
name=Action,
placement=pbth,
within=chapter,
%chapterlistsgaps=off,
]{riak_action}

\newenvironment{actionbox}[3][width=\textwidth]{%
	\begin{figure}
		\includegraphics[#1]{#3}
		\caption[#2]{#2, see Action \ref{action:#3}}.
		\label{fig:#3}
	\end{figure}
	\begin{actionlist}{#2}{#3}
}
{
		\end{actionlist}
}

\newtheorem{hypothesis}{Hypothesis}

\newadjustboxenv{actionlist}[2]{minipage=\textwidth-2\fboxsep-2\fboxrule, fbox, captionbelow={#1, see Figure \ref{fig:#2}}, label={action:#2}, nofloat=riak_action}

\setlength{\belowcaptionskip}{10pt plus 2pt minus 2pt}

\crefname{actionlist}{action}{actions}
\crefname{hypothesis}{hypothesis}{hypotheses}