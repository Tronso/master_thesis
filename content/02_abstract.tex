%!TEX root=../master_thesis.tex

\clearpage
\begin{minipage}{\linewidth}
\section*{Abstract}
\begin{abstract}
Riak Core is a framework to build distributed systems based on the Dynamo architecture.
It uses Consistent Hashing to map keys to nodes as well as to determine the placement of replicas of those keys.
The usage of Consistent Hashing brings some drawbacks like a fixed ring size leading to reduced scalability.
An alternative approach to map keys to nodes is Random Slicing.

We analyze the current state of Riak Core Lite and how Random Slicing can be integrated.
Based on those findings we implement a prototype variant of Riak Core Lite with Random Slicing and conduct an evaluation via benchmarks to compare the variants.
We show that even in a prototype implementation Random Slicing performs on the same levels as Consistent Hashing.
As this shows that Random Slicing is a promising alternative to Consistent Hashing we propose further changes to Riak Core Lite that can raise the performance above its current levels.
\end{abstract}
\begin{otherlanguage}{ngerman}
\section*{Zusammenfassung}
\begin{abstract}
Riak Core ist ein Framework zur Erstellung verteilter Systeme, die auf der Dynamo Architektur basieren.
Es benutzt Consistent Hashing, um sowohl Schlüssel auf Knoten abzubilden, als auch zur Bestimmung der Replikationsplatzierung dieser Knoten.
Die Nutzung von Consistent Hashing bringt einige Nachteile wie eine feste Ringgröße mit sich, die zu reduzierter Skalierbarkeit führen.
Eine alternative Methode, um Schlüssel auf Knoten abzubilden ist Random Slicing.

Wir analysieren Riak Core auf dem momentanen Stand und wie Random Slicing integriert werden kann.
Auf der Grundlage dieser Ergebnisse implementieren wir eine Prototypvariante von Riak Core Lite mit Random Slicing und führen eine Auswertung mithilfe von Benchmarks durch, um die Varianten zu vergleichen.
Wir zeigen, dass schon in einer Prototyp-Implementierung Random Slicing Leistungen auf dem gleichen Niveau wie Consistent Hashing zeigt.
Da dies zeigt, dass Random Slicing eine vielversprechende Alternative zu Consistent Hashing ist, schlagen wir weitere Änderungen an Riak Core Lite vor, die die Leistung über das momentane Level heben können.
\end{abstract}
\end{otherlanguage}
\end{minipage}